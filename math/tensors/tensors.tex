
\title{Tensors }
%\subtitle{Subtitle}
\author{Lambor Marbanaing}
\date{Updated : 05 12 2020}
\maketitle
\tableofcontents

\section{Object}
	\begin{frame}{Object}
		\begin{itemize}
			
			\item Tensors are objects. An object is a some physical quantity that you can say exists in real life. Examples are forces, stresses etc.
			
			\item Each tensor has a co-ordinate frame where you can describe it. For eg : You would need to define a vector by how much it goes in unit basis directions like $e^1, e^2, e^3$. So we can say that a vector may be defined as (3,4,5) describing how we move in $e^1, e^2, e^3$ to represent the vector in the basis and so on. 
		\end{itemize}
	\end{frame}
	
\section{1st order tensors: Vectors}
	\begin{frame}{Vectors}
		\begin{itemize}
			\item A vector can be representated by some components. These components are related with a basis.
			\begin{equation}
			   \ve{v} = v_i e_i 
			\end{equation}
			\item The basis is defined by the right hand rule
			\item  Main operations in vectors are 
				\begin{itemize}
					\item Scalar product or the dot product
					\item Cross product
					\item Vector basis transformation
					
				\end{itemize}
		\end{itemize}
	\end{frame}


	\begin{frame}{Vectors: Dot product}
		\begin{itemize}
			\item The dot product is defined as such 
			\begin{equation}
			\begin{aligned}
				\ve{v.u} = v_ie_i . u_je_j = v_iu_j~e_i.e_j = v_iu_i \delta_{ij} = v_iu_i \\ or \\
				v = (v_1e_1 + v_2e_2 + v_3e_3)(u_1e_1 + u_2e_2 + u_3e_3)\\
			\end{aligned}
			\end{equation}

			\item $e_1,e_2,e_3$ are orthogonal to each other therefore $\delta_{ij}  = 1,0$ 
		\end{itemize}
	\end{frame}


	\begin{frame}{Vectors: Change of basis}
	\begin{itemize}
		\item As explained, every vector is a physical object but the way we define it depends on us.
		\item The way a vector is defined is with respect to the components for any basis. For some basis $\ve{e^1, e^2, e^3}$, we can write as follows
		\begin{align*}
		\ve{v} = \mat{\ve{e_1~e_2~e_3}}\mat{v_1;v_2;v_3} 
		\end{align*}
		\item Therefore in some other basis the vector can be defined with different components as : 
		\begin{align*}
		\ve{v} =\mat{\ve{e'_1~e'_2~e'_3}}\mat{v'_1;v'_2;v'_3} = \ve{Q \mat{v'_1;v'_2;v'_3}}		
		\end{align*}
		\item Therefore $\mat{v'_1;v'_2;v'_3} = \ve{Q^{-1}}\mat{v_1;v_2;v_3}$
		
 
	\end{itemize}
	\end{frame}


	\begin{frame}
		\begin{itemize}
			\item $\ve{Q}$ is the position of the new basis with respect to the old one. That's why we can operate $\ve{Q}$ on $\ve{v}$
			\begin{equation}
				\ve{Q} = \mat{e1.e1'~e1.e2'~e1.e3';e2.e1'~e2.e2'~e2.e3';e3.e1'~e3.e2'~e3.e3'}
			\end{equation}
			\item The first column, gives the location of e1 with respect to the old basis. As we can see it gives the direction cosines. And so on
			\item If we do choose the new basis an orthogonal basis, we get $\ve{Q^{-1} = Q^T}$
			\item $[v]' = \ve{Q^T}[v]$
			\item $[v] = \ve{Q}[v]'$
			\item $[]$ denotes that we are working with only the components, in basic matrix form
		\end{itemize}
	\end{frame}

\section{2nd order tensors : (Many types of objects!)}

	\begin{frame}
		\begin{itemize}
			\item A second order tensor can be a linear map from one vector to another vector
			\item It can be a mapping that takes two vectors and gives a scalar			
			\item But for now we'll focus mainly on the first thing
		\end{itemize}
	
		\begin{block}{Linear map $\ve{S}$}
			$\ve{v = Su}$
		\end{block}
	\end{frame}

	\begin{frame}
		content
	\end{frame}