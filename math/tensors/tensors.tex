
\title{Tensors }
%\subtitle{Subtitle}
\author{Lambor Marbanaing}
\date{Updated : 8 Dec}
\maketitle
\tableofcontents

\section{Object}
	\begin{frame}{Object}
		\begin{itemize}
			
			\item Tensors are objects. An object is a some physical quantity that you can say exists in real life. Examples are forces, stresses etc.
			
			\item Each tensor has a co-ordinate frame where you can describe it. For eg : You would need to define a vector by how much it goes in unit basis directions like $e^1, e^2, e^3$. So we can say that a vector may be defined as (3,4,5) describing how we move in $e^1, e^2, e^3$ to represent the vector in the basis and so on. 
		\end{itemize}
	\end{frame}
	
\section{1st order tensors: Vectors}
	\begin{frame}{Vectors}
		\begin{itemize}
			\item A vector can be representated by some components. These components are related with a basis.
			\begin{equation}
			   \ve{v} = v_i e_i 
			\end{equation}
			\item The basis is defined by the right hand rule
			\item  Main operations in vectors are 
				\begin{itemize}
					\item Scalar product or the dot product
					\item Cross product
					\item Vector basis transformation
					
				\end{itemize}
		\end{itemize}
	\end{frame}


	\begin{frame}{Vectors: Dot product}
		\begin{itemize}
			\item The dot product is defined as such 
			\begin{equation}
			\begin{aligned}
				\ve{v.u} = v_ie_i . u_je_j = v_iu_j~e_i.e_j = v_iu_i \delta_{ij} = v_iu_i \\ or \\
				v = (v_1e_1 + v_2e_2 + v_3e_3)(u_1e_1 + u_2e_2 + u_3e_3)\\
			\end{aligned}
			\end{equation}

			\item $e_1,e_2,e_3$ are orthogonal to each other therefore we get $v_1u_1 + v_2u_2+v_3u_3$
		\end{itemize}
	\end{frame}


	\begin{frame}{Vectors: Change of basis}
	\begin{itemize}
		\item As explained, every vector is a physical object but the way we define it depends on us.
		\item The way a vector is defined is with respect to the components for any basis. For some basis $\ve{e^1, e^2, e^3}$, we can write as follows
		\begin{align*}
		\ve{v} = \mat{\ve{e_1~e_2~e_3}}\mat{v_1;v_2;v_3} 
		\end{align*}
		\item Therefore in some other basis the vector can be defined with different components as : 
		\begin{align*}
		\ve{v} =\mat{\ve{e'_1~e'_2~e'_3}}\mat{v'_1;v'_2;v'_3} = \ve{Q \mat{v'_1;v'_2;v'_3}}		
		\end{align*}
		\item Therefore $\mat{v'_1;v'_2;v'_3} = \ve{Q^{-1}}\mat{v_1;v_2;v_3}$
		
 
	\end{itemize}
	\end{frame}


	\begin{frame}
		\begin{itemize}
			\item $\ve{Q}$ is the position of the new basis with respect to the old one. That's why we can operate $\ve{Q}$ on $\ve{v}$
			\begin{equation}
				\ve{Q} = \mat{e1.e1'~e1.e2'~e1.e3';e2.e1'~e2.e2'~e2.e3';e3.e1'~e3.e2'~e3.e3'}
			\end{equation}
			\item The first column, gives the location of e1' with respect to the old basis. As we can see it gives the direction cosines. And so on
			\item If we do choose the new basis an orthogonal basis, we get $\ve{Q^{-1} = Q^T}$
			\item $[v]' = \ve{Q^T}[v]$
			\item $[v] = \ve{Q}[v]'$
			\item $[]$ denotes that we are working with only the components, in basic matrix form
			\item ?????????????????????????KEEEEEP FIGURE?????????????????????????
		\end{itemize}
	\end{frame}


	\begin{frame}{Basis transformation}
		Let's look at the transformation $\ve{Q}$
		
		\begin{itemize}
			\item We will show that  $\ve{e_i' = Q e_i}$ for $i=1,2,3$
			\begin{equation}
			\ve{Q} = \mat{e1.e1'~e1.e2'~e1.e3';e2.e1'~e2.e2'~e2.e3';e3.e1'~e3.e2'~e3.e3'}
			\end{equation}
			We have said that each column represents the location of the new basis to the original basis
			
			\item $\ve{Qe_1 = Q\mat{1;0;0} = \mat{e1.e1';e2.e1';e3.e1'}}$ and so on
			\item Therefore $\ve{e_i' = Q e_i}$
			\item $\ve{Q^T}$ is the opposite transformation. All with respect to and in the same original basis. 
		\end{itemize}
	\end{frame}

\section{2nd order tensors : (Many types of objects!)}

	\begin{frame}
		\begin{itemize}
			\item A second order tensor can be a linear map from one vector to another vector
			\item It can be a mapping that takes two vectors and gives a scalar			
			\item But for now we'll focus mainly on the first thing
		\end{itemize}
	
		\begin{block}{Linear map $\ve{S}$}
			$\ve{v = Su}$
		\end{block}
	
	\end{frame}

	\begin{frame}
		\begin{itemize}
			\item The map is linear because it is a function that is linear with respect to the vectors it acts on ????FIG????
			\item $\ve{S(\alpha u_1 + \beta u_2) = \alpha Su_1 + \beta  Su_2}$
			
			\item Now $\ve{u}$ could have been defined with respect to any basis. Therefore the linear map as an object is physical.
			
			\item The linear map however has it's own components (Think the values in a matrix) that are defined in a certain basis itself. 
			
			\item Changing the vector basis, means that the linear map should also have its components defined in that basis, for it to do the same transformation. Remember the tensor should do the same thing in any basis! 			
		\end{itemize}
	\end{frame}


	\begin{frame}{Tensor properties}
		\begin{itemize}
			\item $\ve{(A+B)v =  Av + Bv}$
			\item $\ve{ABv = A(Bv)}$
			\item $\ve{A^{-1}A = I}$
			\item $\ve{u.Sv = v.S^Tu}$  ; Now this can be written as $\ve{Sv}$ gives a vector which is then dot producted with $\ve{u}$ 
			
			$\ve{u . (Sv) = \mat{u_1~u_2~u_3} \mat{S_{1j}v_j;S_{2j}v_j;S_{3j}v_j}}$
			
			\item $\ve{u.v = v.u}$ (Scalar value does not change
			)
			\item $\ve{S^{T} =  S}$ (Symmetric)
			\item $\ve{S^{T} =  -S}$ (Skew-symmetric) : Eg $\ve{W_w u = w x u}$ \footnote{Check Bonet Page : 30}
		\end{itemize}
	\end{frame}

	\begin{frame}{Decomposition}
		\begin{itemize}
			\item $\ve{A = S + W }$, $\ve{S,W}$ are symmetric and skew symmetric
			\item $\ve{A = QS }$, $\ve{S,Q}$ are symmetric and orthogonal tensor\\
			
			In first one:
			\begin{itemize}
				\item $\ve{S = \left(A+A^T \right)/2}$
				\item   $\ve{W =  \left(A-A^T \right)/2}$				
			\end{itemize}
			
			In the second one it's called the polar decomposition
		\end{itemize}
	
		\begin{block}{Importance}
			\begin{itemize}
				\item  as we can see how a stress tensor can be decomposed to its hydrostatic (Only axial) and distortion part(Shearing part)
				\item The deformation of a body can also be thought about how it rotates ($\ve{Q}$) and stretches ($\ve{S}$) the body
				
			\end{itemize}
		\end{block}
	\end{frame}


	\begin{frame}{Tensor basis}
		\begin{itemize}
			\item  Just  like a vector is defined as $\ve{v = \mat{v_1;v_2;v_3}}$. These are just components in a basis given as $\ve{v = v_1\mat{1;0;0}+v_2\mat{0;1;0}+v_3\mat{0;0;1}}$
			\item So when we write a tensor in matrix form $\ve{S = \mat{a,b,c;d,e,f;g,h,i}}$ means that these are components in some basis. In choosing this basis we use a product called dyads or tensor product
			\item $\ve{u \otimes v} = \mat{u_1;u_2;u_3}\mat{v_1,v_2,v_3}$
		\end{itemize}
	\end{frame}

	\begin{frame}
		
	\end{frame}