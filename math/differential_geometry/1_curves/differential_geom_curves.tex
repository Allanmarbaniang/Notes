% Full instructions available at:
% https://github.com/elauksap/focus-beamertheme

\documentclass[9pt]{beamer}
\usetheme{focus}

%%%%%%%%%%%%%%%%%%%%%%%%%%%%%%%%%%%%%%%%%%%%%%%%%%%%%%%%%%%%%%%%%%%%%
% Typography, change document font
\usepackage[tt=false, type1=true]{libertine}
\usepackage[varqu]{zi4}
\usepackage[libertine]{newtxmath}
\usepackage[T1]{fontenc}

\usepackage[protrusion=true,expansion=true]{microtype}

% Disable paragraph indentation, and increase gap
\usepackage{parskip}

%Matrix
\usepackage{tabstackengine}
\setstackEOL{;}% row separator
\setstackTAB{,}% column separator
\setstacktabbedgap{1ex}% inter-column gap 
\setstackgap{L}{1.0\normalbaselineskip}% inter-row baselineskip
\let\mat\bracketMatrixstack

\newcommand{\pth}{Figure/}
\newcommand{\ve}[1]{\mathbf{#1}}

% Copyright (C) 2018-2019 Pasquale Claudio Africa and the LaTeX community.
% A full list of contributors can be found at
%
%     https://github.com/elauksap/focus-beamertheme
% 
% This file is part of beamerthemefocus.
% 
% beamerthemefocus is free software: you can redistribute it and/or modify
% it under the terms of the GNU General Public License as published by
% the Free Software Foundation, either version 3 of the License, or
% (at your option) any later version.
% 
% beamerthemefocus is distributed in the hope that it will be useful,
% but WITHOUT ANY WARRANTY; without even the implied warranty of
% MERCHANTABILITY or FITNESS FOR A PARTICULAR PURPOSE. See the
% GNU General Public License for more details.
% 
% You should have received a copy of the GNU General Public License
% along with beamerthemefocus. If not, see <http://www.gnu.org/licenses/>.

\mode<presentation>


% DEFINE COLORS. ---------------------------------------------------------------
\definecolor{main}{RGB}{134, 161, 174}
\definecolor{main2}{RGB}{104, 131, 144}
\definecolor{textc}{RGB}{20, 20, 20}
\definecolor{background}{RGB}{255, 255, 255}

\definecolor{alert}{RGB}{180, 0, 0}
\definecolor{example}{RGB}{0, 110, 0}


% SET COLORS. ------------------------------------------------------------------
\setbeamercolor{normal text}{fg=textc, bg=background}
\setbeamercolor{alerted text}{fg=textc}
\setbeamercolor{example text}{fg=textc}

\setbeamercolor{titlelike}{fg=background, bg=main}
\setbeamercolor{frametitle}{parent={titlelike}}

\setbeamercolor{footline}{fg=background, bg=main2}

\setbeamercolor{block title}{bg=main!80!background, fg=background}
\setbeamercolor{block body}{bg=main!10!background, fg=textc}

\setbeamercolor{block title alerted}{bg=alert, fg=background}
\setbeamercolor{block body alerted}{bg=alert!10!background, fg=textc}

\setbeamercolor{block title example}{bg=example, fg=background}
\setbeamercolor{block body example}{bg=example!10!background, fg=textc}

\setbeamercolor{itemize item}{fg=textc}
\setbeamercolor{itemize subitem}{fg=textc}

\setbeamercolor{enumerate item}{fg=textc!70!black}
\setbeamercolor{enumerate subitem}{fg=textc!70!black}

\setbeamercolor{description item}{fg=textc!70!black}
\setbeamercolor{description subitem}{fg=textc!70!black}

\setbeamercolor{caption name}{fg=textc}

\setbeamercolor{section in toc}{fg=textc}
\setbeamercolor{subsection in toc}{fg=textc}
\setbeamercolor{section number projected}{bg=textc}
\setbeamercolor{subsection number projected}{bg=textc}

\setbeamercolor{bibliography item}{fg=main}
\setbeamercolor{bibliography entry author}{fg=main!70!black}
\setbeamercolor{bibliography entry title}{fg=main}
\setbeamercolor{bibliography entry location}{fg=main}
\setbeamercolor{bibliography entry note}{fg=main}

\mode<all>

\title{Differential Geometry  }
%\subtitle{Subtitle}
\author{Allan}
\date{Updated : \today}


\begin{document}
\maketitle
\tableofcontents	

\section{Differential geometry : Curved computing}


	\begin{frame}
		\begin{itemize}
			\item A shell is comprised of its reference surface, thickness and edges. The thickness of the shell at a point is found as the distance between the bounding surface (Top and bottom surfaces) as measured along a normal to the reference surface that passes through the point.  
			\item FIX ME???????????? New things also have to be added
		\end{itemize}
	\end{frame}


	\begin{frame}{Space curves : Parameterisation of curve}
		\begin{itemize}
			\item A 3d curve in a rectangular coordinate system ($x_1,x_2,x_3$) can be represented by the locus of the end point of the position vector
			\begin{equation}
				\ve{x =} x_1(t)\ve{e_1} + x_2(t)\ve{e_2} + x_3(t)\ve{e_3} 
			\end{equation}
			for all the values of the parameter $t$ that are in the interval $t_1<t<t_2$. Suppose that $\ve{x}$ takes only  a single value of the parameter $t$, then we insure it's uniqueness. 
		\end{itemize}
	\end{frame}


	\begin{frame}{Unit tangent vector}
		\begin{itemize}
			\item We take $s$ as the variable of the arc length along the space curve. The derivative of the position vector $x$ with respect to $s$ is given as
			\begin{equation}
				\frac{d\ve{x}}{ds} = \frac{dx_1}{ds}\ve{e_1} +\frac{dx_2}{ds}\ve{e_2}+\frac{dx_3}{ds}\ve{e_3}
			\end{equation}
			\item If we form a scalar product with itself we get
			\begin{equation}
				\frac{d \ve{x}}{ds} \cdot \frac{d \ve{x}}{ds} = 
				\left(\frac{dx_1}{ds} \right)^2 + \left(\frac{dx_2}{ds} \right)^2 + \left(\frac{dx_3}{ds} \right)^2
			\end{equation}
			\item From differential calculus we know that
			\begin{equation}
				\left(ds \right)^2 = \left(dx_1 \right)^2 + \left(dx_2 \right)^2 + \left(dx_3 \right)^2
			\end{equation}
			Hence
			\begin{equation}
				\frac{d \ve{x}}{ds} \cdot \frac{d \ve{x}}{ds} =  1
			\end{equation}
			This confuese me how the magnitude of the tangent becomes 1, because sometimes in calculus some mainpulations seem very devious
		\end{itemize}
	\end{frame}


	\begin{frame}
		\begin{itemize}
			\item We can think of a vector $\Delta\ve{ x}$ joining two points Q and Q' on a curve. The vector $\Delta\ve{ x}/\Delta s$ has the same directiona as $\Delta\ve{ x}$ and as $\Delta s$ approaches 0, $\Delta\ve{ x}/\Delta s$ becomes the tangent vector to the curve at he poin Q. 
			\item The vector $\ve{T}$ is as the limit of $\Delta s \rightarrow 0$ we get 
			\begin{equation}
				\ve{T} = \frac{d \ve{x}}{ds} 
			\end{equation}
			\item We also note that 
			\begin{equation}
				{{x}'} = \frac{d\ve{x}}{dt} = \frac{d\ve{x}}{ds}\frac{ds}{dt} 
			\end{equation}
			This comes from chain rule by the way the parametrisation is done orginally. This is als a tangent vector but it is not of unit length
		\end{itemize}
	\end{frame}


	\begin{frame}{Osculating plane, Principla Normal}
		\begin{itemize}
			\item The tangent to the Curve at Q is found to be the limiting position of the line connecting the points Q and Q'. Now if we consider the limiting position of a plane passing through three consequitve places. We will get the osculating plane joining the normal and the tangent of a curve.
			\item The tripple scalar produc of three coplanar vectors is zero and the expression of the osculating plane is as
			\begin{equation}
			\ve{(X-x)\cdot ({x}' \times {x}'')}
			\end{equation}
			
		\end{itemize}
	\end{frame}


	\begin{frame}{Curvature}
		\begin{itemize}
			\item By taking the dot product with itself $\ve{T \cdot T = 1}$. If we differentiate this scalar product with respect to the arc length we get 
			\begin{equation}
			\frac{d}{ds}(\ve{T\cdot T}) =  2 \ve{T \cdot T}^o = 0
			\end{equation}
			where $\ve{T}^o$ is the derivative with respect to s. Now this tells us that the tangent is prependicular to the curvature. We can also then write
			\begin{equation}
				\ve{T} = \frac{d\ve{x}}{ds} = \frac{d\ve{x}}{dt} \frac{dt}{ds} = \ve{x'}t^o 
			\end{equation} 
			where this comes from chain rule but the parameterisation is messed up, originally it is $\frac{d\ve{x}}{dt} = \frac{d\ve{x}}{ds}\frac{ds}{dt} $. I don't know when things can be used like fractions!! But it checks out here. Also note that s and t will be different functions for different components of $\ve{x}$
			\item Therefore we can say 
			\begin{equation}
			\ve{T^o = x'}t^{oo} + \ve{x''}(t^o)^2
			\end{equation}
			Which shoes that the curvature also lies in the plane of $\ve{x'}$ and $\ve{x''}$, or the osculating plane.
		\end{itemize}
	\end{frame}


	\begin{frame}
		\begin{itemize}
			\item Since the curvature has been shown to be perpendicular to the tangent $\ve{T}$ we can say that the curvature is parallel to the principal normal given by
			\begin{equation}
				\ve{T^o} = k \ve{N} = \ve{k}
			\end{equation}
			where $\ve{N}$ is a unit normal vector in the direction of the principal normal to the curve at a point (1/R). We can see that, k is the curvature and $k\ve{\ve{N}}$ is the curvature vector.
			\item Although the sense of $T^o$ is determined solely by the curve, the sense of the principal normal $\ve{N }$ is arbitary. k therefore depends on the sense of $\ve{N}$.
	\end{itemize}
	\end{frame}

\end{document}