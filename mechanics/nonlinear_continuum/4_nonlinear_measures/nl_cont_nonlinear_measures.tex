\input{../../../temp/preamble}
% Copyright (C) 2018-2019 Pasquale Claudio Africa and the LaTeX community.
% A full list of contributors can be found at
%
%     https://github.com/elauksap/focus-beamertheme
% 
% This file is part of beamerthemefocus.
% 
% beamerthemefocus is free software: you can redistribute it and/or modify
% it under the terms of the GNU General Public License as published by
% the Free Software Foundation, either version 3 of the License, or
% (at your option) any later version.
% 
% beamerthemefocus is distributed in the hope that it will be useful,
% but WITHOUT ANY WARRANTY; without even the implied warranty of
% MERCHANTABILITY or FITNESS FOR A PARTICULAR PURPOSE. See the
% GNU General Public License for more details.
% 
% You should have received a copy of the GNU General Public License
% along with beamerthemefocus. If not, see <http://www.gnu.org/licenses/>.

\mode<presentation>


% DEFINE COLORS. ---------------------------------------------------------------
\definecolor{main}{RGB}{134, 161, 174}
\definecolor{main2}{RGB}{104, 131, 144}
\definecolor{textc}{RGB}{20, 20, 20}
\definecolor{background}{RGB}{255, 255, 255}

\definecolor{alert}{RGB}{180, 0, 0}
\definecolor{example}{RGB}{0, 110, 0}


% SET COLORS. ------------------------------------------------------------------
\setbeamercolor{normal text}{fg=textc, bg=background}
\setbeamercolor{alerted text}{fg=textc}
\setbeamercolor{example text}{fg=textc}

\setbeamercolor{titlelike}{fg=background, bg=main}
\setbeamercolor{frametitle}{parent={titlelike}}

\setbeamercolor{footline}{fg=background, bg=main2}

\setbeamercolor{block title}{bg=main!80!background, fg=background}
\setbeamercolor{block body}{bg=main!10!background, fg=textc}

\setbeamercolor{block title alerted}{bg=alert, fg=background}
\setbeamercolor{block body alerted}{bg=alert!10!background, fg=textc}

\setbeamercolor{block title example}{bg=example, fg=background}
\setbeamercolor{block body example}{bg=example!10!background, fg=textc}

\setbeamercolor{itemize item}{fg=textc}
\setbeamercolor{itemize subitem}{fg=textc}

\setbeamercolor{enumerate item}{fg=textc!70!black}
\setbeamercolor{enumerate subitem}{fg=textc!70!black}

\setbeamercolor{description item}{fg=textc!70!black}
\setbeamercolor{description subitem}{fg=textc!70!black}

\setbeamercolor{caption name}{fg=textc}

\setbeamercolor{section in toc}{fg=textc}
\setbeamercolor{subsection in toc}{fg=textc}
\setbeamercolor{section number projected}{bg=textc}
\setbeamercolor{subsection number projected}{bg=textc}

\setbeamercolor{bibliography item}{fg=main}
\setbeamercolor{bibliography entry author}{fg=main!70!black}
\setbeamercolor{bibliography entry title}{fg=main}
\setbeamercolor{bibliography entry location}{fg=main}
\setbeamercolor{bibliography entry note}{fg=main}

\mode<all>


\begin{document}
	\tableofcontents
	\section{Non-Linear strain measures introduction : \today}

	\begin{frame}{Introduction}
		\begin{itemize}
			\item Now suppose of being rigid, the body is deformable that is the relative deformation is introducing strain and stresses
			\item LARGE DISPLACEMENTS + LARGE STRAINS
			\item The first deals with displacements that are large, and therefore while finding the strains, we have to use higher orders of the displacement derivatives
			\item The second deals with ????????????????????????????		
		\end{itemize}
	\end{frame}

\section{OneD measures}
	\begin{frame}{One-D strain measures}
		\begin{itemize}
			\item Emphasize its only for One D! But same theory for other dimensions
			\item A strain measure need not be fixed. Sometimes the strain measure we usually use may not be able to model the correct behaviour. When we choose any strain measure, the proper corresponding stress and the  constitutive relationship ($\ve{\sigma = C \varepsilon}$) has to be taken. 
			\item The stress and strain have to be "work compatible". That is they are together used in the strain energy density function. 			
		\end{itemize}
	\end{frame}

	\begin{frame}{One-D strain measures : Types}
		\begin{block}{Engineering strain}
				\begin{itemize}
				\item Engineering strain $\varepsilon_E = \dfrac{l - L}{L} =\dfrac{\Delta }{L}$. l is deformed length, L is initial undeformed
				\item We could have also divided $\Delta$ by l (Change by deformed length). If $l \approx L$ then it would not matter.
				\item $\varepsilon_E $ is the small infinitesimal strain, where the deformed and undeformed lengths are very similar.			
			\end{itemize}
		\end{block}
		\begin{block}{Logarithimic strain}
			\begin{itemize}
			\item The instantaneous strain increment can be thought as $\varepsilon_L = \frac{\Delta_1}{L} + \frac{\Delta_2}{l_1} ...$
			\item Or $d\varepsilon_L = \frac{dl}{l}$  
			\item $\varepsilon_L = \int_{L}^{l}\dfrac{dl}{l} = ln \frac{l}{L}$
			\item The integration is done between two configurations $L \rightarrow l$
			\end{itemize}
		\end{block}	
	\end{frame}

	\begin{frame}{One-D strain measures : Types}
		These strains are more easily extrapolated to continuum (3d cases)
		\begin{block}{Green strain}
			\begin{itemize}
				\item $\varepsilon_G = \dfrac{l^2 - L^2}{2L^2}$ 			
			\end{itemize}
		\end{block}
		\begin{block}{Almansi strain}
			\begin{itemize}
				\item $\varepsilon_A = \dfrac{l^2 - L^2}{2l^2}$ 
			\end{itemize}
		\end{block}	
	
	\begin{itemize}
		\item Suppose $l \approx L$ and therefore $\Delta$ is small
		\item And $l = (L + \Delta)$
		\item $\varepsilon_G = \dfrac{(L + \Delta)^2 - L^2}{2L^2} = \dfrac{(L^2 + \Delta^2 + 2 L \Delta  - L^2)}{2L^2} \approx \dfrac{\Delta }{L}$ (As $\Delta$ is very small and so $\Delta^2$ vanishes)
		
	\end{itemize}
	\end{frame}

	\begin{frame}{Problem \#1}
	\begin{figure}
		\centering
		\includegraphics[width=0.6\linewidth]{Figure/fig4}
		\caption{}
		\label{fig:fig1}
	\end{figure}
	\begin{itemize}
		\item Initial length L, area A, volume V
		\item  Final length l, area a, volume v
		
	\end{itemize}
	\end{frame}

	\begin{frame}
		\begin{itemize}
			\item In defining the equilibrium (Froces = 0 , No moments). We will be defining the internal stress by different strain measures.
			\item Remember that a proper constitutive law has to be taken for a particularly strain measure
			\item Here we have chosen the Cauchy stress and E randomly and not dependant on work compatibility. The cauchy stress is the actual/true stress in the deformed state. (Or it is the stress in the deformed state which is in equilibrium)
			
		\end{itemize}
		\begin{block}{Using two strain measures}
			Green and logarithmic.
			\begin{itemize}
				\item Cauchy stress (True stress) $\sigma = E \varepsilon$ can be :
				\item $\sigma = E \dfrac{l^2-L^2}{L^2}$
				\item $\sigma = E ln\dfrac{l}{L}$
			\end{itemize}
		\end{block}
	\end{frame}

	\begin{frame}
		\begin{figure}
			\centering
			\includegraphics[width=0.3\linewidth]{Figure/fig4}
			\label{fig:fig1}
		\end{figure}
		\begin{itemize}
		\item The bar will keep moving up until the vertical equilibrium is reached.
		\item Vertical equilibrium at B is $F - T(x)sin \theta(x) = 0$, where $T(x)$ is the internal force and depends on $x$. $\theta$ is also dependant on $x$
		\item Now we can construct a residual function $R(x) =  F - T(x)sin \theta(x)$ where the residual becomes zero for a particular solution of x. \footnote{Note that $\frac{d R}{d x}$ is the tangent stifness $K_{Bx}$ or force in direction B due to displacement x.}  So
		\begin{equation}
		R(x) = \sigma a sin \theta - F = \sigma(x) a \frac{ x}{l} - F
		\end{equation} 	
		\end{itemize}
		
		\begin{block}{Stress dependant on different strain measures}
			\begin{itemize}
				\item $T(x) = E \dfrac{l^2-L^2}{L^2} a \dfrac{x}{l}$  \hfill ($\sigma =  E \varepsilon_G$)
				\item $T(x) = E ln\dfrac{l}{L} a \dfrac{x}{l}$ \hfill	($\sigma =  E \varepsilon_L$)			
			\end{itemize}
		\end{block}
	\end{frame}

	\begin{frame}
		
	\begin{block}{Stress dependant on different strain measures}
		\begin{itemize}
			\item $T(x) = E \dfrac{l^2-L^2}{L^2} a \dfrac{x}{l}$  \hfill ($\sigma =  E \varepsilon_G$)
			\item $T(x) = E ln\dfrac{l}{L} a \dfrac{x}{l}$ \hfill	($\sigma =  E \varepsilon_L$)			
		\end{itemize}
	\end{block}

	\begin{itemize}
		\item l is a function of x, $l^2=D^2+x^2$
		\item R(x) is therefore very nonlinear with respect to x. In R(x), F is not dependant on x. But sometimes it can be the case that the load is also nonlinear.	
	\end{itemize}
	
	\begin{block}{Solving}
		We need to solve the nonlinear equation R(x) = 0
		
		\begin{itemize}
			\item So we use NR, or first order taylor series to linearise R and solve it iteratively
			
			\item $R(x_{i+1}) = R(x_{i}) + \frac{dR}{dx}|_{x_{i}}(x_{i+1}-x{i})$
			\item We want R = 0 , so the value $R(x_{i+1}) = 0$
			\item  $0 = R(x_{i}) + \frac{dR}{dx}|_{x_{i}}(x_{i+1}-x{i})$		
		\end{itemize}
	\end{block}
	\end{frame}

	\begin{frame}
		\begin{figure}
			\centering
			\includegraphics[width=1\linewidth]{Figure/fig5}
			\label{fig:fig1}
		\end{figure}
	\end{frame}

	\begin{frame}{Summary Problem \#1}
		\begin{itemize}
			\item Regions where x is small, different mesasures gives okay results
			\item We see different behaviour behaviours between the strain measures at higher strain
			\item Snap through behaviour if we increase compressive load too much. Imagine you are pushing the truss down (-x) and suddenly it will roll to the other side. 
			\item If truss  is initially vertical  (Like a column in tension, Therefore no rotation) : Same E should have not been used for both different strain measures. (It seems that the green strain looks good as we expect it to be linear in axial)
			\item Initially horizontal : Stiffening due to tension
			
		\end{itemize}
	\end{frame}

	\begin{frame}{Further insight}
	\begin{itemize}
		\item A comment was made that E should not have been used. 
		\item The vertical stiffness $K_{Bx}$ is the chagne in equilibrium at B in direction x. $K_{Bx} =  \frac{dR}{dx}$. If F is constant then $\frac{dR}{dx} =  \frac{dT}{dx}$
		\item Without the inclusion of the strain measures, internal force is $T(x) = \sigma a \frac{ x}{l} $. (All three are a function of x)
		\item Since both strain measures are function of $l$ we can write $\sigma = f(l)$
		\item Using the incompressibility codition \footnote{The condition states that volume cant change under deformation and so $al = AL$}, we can replace $a$ with $a =  \frac{V}{l}$, and using chain rule:
		\begin{align*}
			 \frac{dT}{dx} = \frac{d}{dx}\left(\dfrac{\sigma V x}{l^2} \right)=\frac{Vx}{l^2}\frac{\partial \sigma}{\partial l} \frac{\partial l}{\partial x} -\frac{2\sigma Vx}{l^3}\frac{\partial l}{\partial x } + \frac{\sigma V}{l^2} \\
			  \frac{dT}{dx} = \frac{ax}{l}\frac{\partial \sigma}{\partial l} \frac{\partial l}{\partial x} -\frac{2\sigma ax}{l^2}\frac{\partial l}{\partial x } + \frac{\sigma a}{l}
		\end{align*}
	\end{itemize}
	\end{frame}

	\begin{frame}
		\begin{align*}
		\frac{dT}{dx} = \frac{ax}{l}\frac{\partial \sigma}{\partial l} \frac{\partial l}{\partial x} -\frac{2\sigma ax}{l^2}\frac{\partial l}{\partial x } + \frac{\sigma a}{l}
		\end{align*}
		\begin{block}{Stress gradient}
			\begin{itemize}
				\item So we need to find $\dfrac{\partial \sigma}{\partial l}$
				
				\item Green : $\left(\dfrac{\partial \sigma}{\partial l} \right)_G = E \dfrac{\partial \varepsilon_G}{\partial l} = E2l/2L^2 = \dfrac{El}{L^2}$
				
				\item Logarithmic : $\left(\dfrac{\partial \sigma}{\partial l} \right)_L = E \dfrac{\partial \varepsilon_L}{\partial l} = E \frac{d}{dl}\left(ln(l) - ln(L) \right) =  \dfrac{E}{l}$		
			\end{itemize}
		\end{block}
		\begin{block}{l gradient}
			\begin{itemize}
				\item $l^2 = D^2 + x^2$	
				\item $2l \dfrac{dl}{dx} = 2x$
				\item $\dfrac{dl}{dx} = \dfrac{x}{l}$
			\end{itemize}
		\end{block}
	\end{frame}

	\begin{frame}{Stiffness}
		\begin{block}{}
			\begin{itemize}
				\item $K_{Bx} =  \frac{dR}{dx} = \frac{dT}{dx}$
				
				\item Green : $K_{G} =  \dfrac{A}{L}\left(E - 2\sigma \dfrac{L^2}{l^2} \right)\dfrac{x^2}{l^2} + \dfrac{\sigma a}{l}$ \footnote{V = AL}
				
				\item  Logirthmic : $K_{L} =  \dfrac{a}{l}\left(E - 2\sigma \right)\dfrac{x^2}{l^2} + \dfrac{\sigma a}{l}$		
			\end{itemize}
		\end{block}
	
		\begin{itemize}
			\item  They look similar but the causal consitutive relation chosen has led to the different results
			
			\item We will write $K_G$ as with the idea of getting an insight:
			
			\begin{align*}
				K_{G} =  \dfrac{A}{L}\left(E - 2S \right)\dfrac{x^2}{l^2} + \dfrac{S A}{l}\\
				\text{where}~S = \sigma \frac{L^2}{l^2}
			\end{align*}
			\item  Where S is the second-Piola Kirchoff stress which gives the force per unit underformed area transformed by the deformation gradient inverse $(l/L)^{-1}$
		\end{itemize}
		 
	\end{frame}

	\begin{frame}
		\begin{itemize}
			\item S is actually associated with $\varepsilon_G$
			\item $(x/l)^2$ is the transformation from local to global forces. 
			\item Therefore $K_G$ shows that we can express the stiffness in initial underformed configuration
			\item If x is close to X and l is close to L then both the stiffness would be the same. The second term contains the change $\frac{\partial l}{\partial x}$, so this term disappears. 
			\item The third term is the initial stress or geometric stiffness. This is unconcerned with the change in cross sectional area and associated only with the change in rigid body rotation. A very negative value can cause instability and singular K. The third term actually came from the derivative of the direction cosines (x/L). 
			
		\end{itemize}
	\end{frame}

\section{Continuum measures - 2D}

	\begin{frame}
		\begin{block}{}
			\begin{itemize}
				\item Strain $\varepsilon$ has components $\varepsilon_x,y,xy$
				 \item This strain is a measurement at a point!!!
				 \item Infinitesimal strains 
				 \begin{align*}
				 	\varepsilon_x = \frac{\partial u}{\partial x} \\
					\varepsilon_y = \frac{\partial v}{\partial y} \\
					\varepsilon_{xy} = \frac{1}{2}\left( \frac{\partial u}{\partial y} + \frac{\partial v}{\partial x}\right) \\
				 \end{align*}
				
			\end{itemize}
		\end{block}
	\begin{itemize}
		\item Displacements are small, so only linear orders of displacement gradients are available
		\item Notation for different configurations undeformed : x, and deformed : X
		
	\end{itemize}
	\end{frame}

	\begin{frame}{Infinitesimal strain problem}
		\begin{figure}
			\centering
			\includegraphics[width=0.4\linewidth]{Figure/fig6}
			\label{fig:fig1}
		\end{figure}
		\begin{block}{}
			\begin{itemize}
				\item Suppose there is a rotation in any solid by 90 $^o$ , No deformation!. So:
				\begin{align*}
					u = -X - Y \\
					v = X - Y 
				\end{align*}
				\item So our infinitesimal strains are :
				\begin{align*}
					\varepsilon_x = \varepsilon_y = -1 \\  \varepsilon_{xy} = 0
				\end{align*}		
				\item So we still get strain, when we should have not
			\end{itemize}
		\end{block}
	\end{frame}

	\begin{frame}{Other continuum strain measures}
		\begin{itemize}
			\item Using the same Green strain, we extend it in some way for 2D
			\item Taking the differential length dS (Undeformed) and ds (Deformed)
			\item Take a small element dX initially parallel to $x$ axis 			
			\begin{align*}
				ds^2 = \left(dX + \frac{\partial u}{\partial X} dX\right)^2 + \left(\frac{\partial v}{\partial X}dX \right)^2 \\				
				E_{xx} = \frac{ds^2-dX^2}{2dX^2} = \frac{1}{2} \left(\left(1 + \frac{\partial u}{\partial X} \right)^2 + \left(\frac{\partial v}{\partial X} \right)^2\right) - 1 \\
			\end{align*}
			Similarly we get the Green strains equations :
			
			
			\item Thse strain components = 0 for the rigid rotation case
			\item Nonlinear strains are better, but they coincide with the infinitesimal strains when x and X are close to each other. \footnote{Here x and X are vectors that define the total position of a body in the deformed and undeformed}
			
		\end{itemize}
	\end{frame}

\end{document}